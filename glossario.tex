%**************************************************************
% Acronimi
%**************************************************************
\renewcommand{\acronymname}{Acronimi e abbreviazioni}
%**************************************************************

\newacronym[description={\glslink{apig}{Application Programming Interface}}]
{api}{API}{Application Programming Interface}
\newglossaryentry{apig}
{
name=\glslink{api}{API},
text= Application Programming Interface,
sort= api,
description={in informatica con il termine \emph{Application Programming Interface API} (interfaccia di programmazione di un'applicazione) si indica ogni insieme di procedure disponibili al programmatore, di solito raggruppate a formare un set di strumenti specifici per l'espletamento di un determinato compito all'interno di un certo programma.
La finalità è ottenere un'astrazione, di solito tra l'hardware e il programmatore o tra software a basso e quello ad alto livello semplificando così il lavoro di programmazione}
}

\newacronym[description={\glslink{mha}{Multi-Head attention}}]
{mha}{MHA}{Multi-Head Attention}
\newglossaryentry{mhag}
{
    name=\glslink{mha}{MHA},
    text= Multi-Head Attention,
    sort= mha,
    description={Multi head attention is .}
}
\newacronym[description={\glslink{slam}{Simultaneous Localization and Mapping}}]
{slam}{SLAM}{Simultaneous Localization and Mapping}
\newglossaryentry{slamg}
{
    name=\glslink{slam}{SLAM},
    text= Simultaneous Localization and Mapping,
    sort= slam,
    description={Slam is a computational problem of constructing or updating a map of an unknown environment while simultaneously keeping track of an agent's location within it.}
}
%**************************************************************
% Glossario
%**************************************************************
%\renewcommand{\glossaryname}{Glossario}



\newglossaryentry{word}
{
    name=\glslink{word}{Word},
    text=word,
    sort=word,
    description={Example of a term in the glossary}
}

\newglossaryentry{vanishing gradient problem}
{
    name=\glslink{vanishing gradient problem}{Vanishing gradient problem},
    text=vanishing gradient problem,
    sort=vanishing gradient problem,
    description={When there are more layers in the network, the value of the product of derivative decreases until at some point the partial derivative of the loss functions approaches a value close to zero, and the partial derivative vanishes.}
}

%\newglossaryentry{}
%{
%    name=\glslink{}{},
%    text=,
%    sort=,
%    description={}
%}