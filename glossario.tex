%**************************************************************
% Acronimi
%**************************************************************
\renewcommand{\acronymname}{Acronimi e abbreviazioni}
%**************************************************************

\newacronym[description={\glslink{apig}{Application Programming Interface}}]
{api}{API}{Application Programming Interface}
\newglossaryentry{apig}
{
name=\glslink{api}{API},
text= Application Programming Interface,
sort= api,
description={in informatica con il termine \emph{Application Programming Interface API} (interfaccia di programmazione di un'applicazione) si indica ogni insieme di procedure disponibili al programmatore, di solito raggruppate a formare un set di strumenti specifici per l'espletamento di un determinato compito all'interno di un certo programma.
La finalità è ottenere un'astrazione, di solito tra l'hardware e il programmatore o tra software a basso e quello ad alto livello semplificando così il lavoro di programmazione}
}

%**************************************************************
% Glossario
%**************************************************************
%\renewcommand{\glossaryname}{Glossario}



\newglossaryentry{repackaging}
{
name=\glslink{repackaging}{Repackaging},
text=repackaging,
sort=repackaging,
description={In ingegneria del software, il termine repackaging indica il processo di creazione del pacchetto d'installazione a seguito di un processo di trasformazione contraria}
}

%\newglossaryentry{}
%{
%    name=\glslink{}{},
%    text=,
%    sort=,
%    description={}
%}