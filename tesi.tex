% I seguenti commenti speciali impostano:
% 1. 
% 2. PDFLaTeX come motore di composizione;
% 3. tesi.tex come documento principale;
% 4. il controllo ortografico italiano per l'editor.

% !TEX encoding = UTF-8
% !TEX TS-program = pdflatex
% !TEX root = tesi.tex
% !TEX spellcheck = it-IT
%!TEX program = xelatex
\documentclass[11pt,                    % corpo del font principale
    a4paper,                 % carta A4
    twoside,                 % impagina per fronte-retro
    openright,               % inizio capitoli a destra
    english,
    ctexart,
]{book}

%**************************************************************
% Importazione package
%************************************************************** 
%\usepackage{amsmath,amssymb,amsthm}    % matematica
\usepackage[UTF8]{ctex}
\usepackage[T1]{fontenc}                % codifica dei font:
% NOTA BENE! richiede una distribuzione *completa* di LaTeX

\usepackage[utf8]{inputenc}             % codifica di input; anche [latin1] va bene
% NOTA BENE! va accordata con le preferenze dell'editor

\usepackage[english]{babel}    % per scrivere in italiano e in inglese;
% l'ultima lingua (l'italiano) risulta predefinita

\usepackage{bookmark}                   % segnalibri

\usepackage{caption}                    % didascalie

\usepackage{chngpage,calc}              % centra il frontespizio

\usepackage{csquotes}                   % gestisce automaticamente i caratteri (")

\usepackage{emptypage}                  % pagine vuote senza testatina e piede di pagina

\usepackage{epigraph}            % per epigrafi

\usepackage{eurosym}                    % simbolo dell'euro

%\usepackage{indentfirst}               % rientra il primo paragrafo di ogni sezione

\usepackage{graphicx}                   % immagini

\usepackage{hyperref}                   % collegamenti ipertestuali

\usepackage[binding=5mm]{layaureo}      % margini ottimizzati per l'A4; rilegatura di 5 mm


\usepackage{microtype}                  % microtipografia

\usepackage{mparhack,fixltx2e,relsize}  % finezze tipografiche

\usepackage{nameref}                    % visualizza nome dei riferimenti                                      

\usepackage[font=small]{quoting}        % citazioni

%\usepackage{subfig}                     % sottofigure, sottotabelle

\usepackage[english]{varioref}          % riferimenti completi della pagina

\usepackage[dvipsnames]{xcolor}         % colori
\usepackage{formattazione}

\usepackage{booktabs}                   % tabelle                                       
\usepackage{tabularx}                   % tabelle di larghezza prefissata                                    
\usepackage{longtable}                  % tabelle su più pagine                                        
\usepackage{ltxtable}                   % tabelle su più pagine e adattabili in larghezza

\usepackage[toc, acronym]{glossaries}   % glossario
% per includerlo nel documento bisogna:
% 1. compilare una prima volta tesi.tex;
% 2. eseguire: makeindex -s tesi.ist -t tesi.glg -o tesi.gls tesi.glo
% 3. eseguire: makeindex -s tesi.ist -t tesi.alg -o tesi.acr tesi.acn
% 4. compilare due volte tesi.tex.

\usepackage[backend=biber,style=numeric-comp,hyperref,backref]{biblatex}
% eccellente pacchetto per la bibliografia;
% produce uno stile di citazione autore-anno;
% lo stile "numeric-comp" produce riferimenti numerici
% per includerlo nel documento bisogna:
% 1. compilare una prima volta tesi.tex;
% 2. eseguire: biber tesi
% 3. compilare ancora tesi.tex.

%**************************************************************
% file contenente le impostazioni della tesi
%**************************************************************

%**************************************************************
% Frontespizio
%**************************************************************

% Autore
\newcommand{\myName}{Xiaowei Wen}
\newcommand{\myTitle}{Transformer applied to Visual Odometry}

% Tipo di tesi                   
\newcommand{\myDegree}{Master degree thesis}

% Università             
\newcommand{\myUni}{Alma Mater Studiorum - University of Bologna}

% Facoltà       
\newcommand{\myFaculty}{Artificial Intelligence}

% Dipartimento
\newcommand{\myDepartment}{Computer Science and Engineering - DISI}

% Titolo del relatore
\newcommand{\profTitle}{Prof.}

% Relatore
\newcommand{\myProf}{ Luigi Di Stefano}

% Luogo
\newcommand{\myLocation}{Bologna}

% Anno accademico
\newcommand{\myAA}{2021-2022}

% Data discussione
\newcommand{\myTime}{06 October 2022}


\addto\captionsenglish{\renewcommand{\lstlistingname}{Code}}

%**************************************************************
% Impostazioni di impaginazione
% see: http://wwwcdf.pd.infn.it/AppuntiLinux/a2547.htm
%**************************************************************

\setlength{\parindent}{14pt}   % larghezza rientro della prima riga
\setlength{\parskip}{0pt}   % distanza tra i paragrafi


%**************************************************************
% Impostazioni di biblatex
%**************************************************************
\bibliography{bibliografia} % database di biblatex 

\defbibheading{bibliography} {
    \cleardoublepage
    \phantomsection 
    \addcontentsline{toc}{chapter}{\bibname}
    \chapter*{\bibname\markboth{\bibname}{\bibname}}
}

\setlength\bibitemsep{1.5\itemsep} % spazio tra entry

\DeclareBibliographyCategory{opere}
\DeclareBibliographyCategory{web}

\addtocategory{opere}{womak:lean-thinking}
\addtocategory{web}{site:agile-manifesto}

\defbibheading{opere}{\section*{Bibliography}}
\defbibheading{web}{\section*{Websites}}


%**************************************************************
% Impostazioni di caption
%**************************************************************
\captionsetup{
    tableposition=top,
    figureposition=bottom,
    font=small,
    format=hang,
    labelfont=bf
}

%**************************************************************
% Impostazioni di glossaries
%**************************************************************
\input{Glossario} % database di termini
\makeglossaries


%**************************************************************
% Impostazioni di graphicx
%**************************************************************
\graphicspath{{images/}} % cartella dove sono riposte le immagini


%**************************************************************
% Impostazioni di hyperref
%**************************************************************
\hypersetup{
    %hyperfootnotes=false,
    %pdfpagelabels,
    %draft,	% = elimina tutti i link (utile per stampe in bianco e nero)
    colorlinks=true,
    linktocpage=true,
    pdfstartpage=1,
    pdfstartview=FitV,
    % decommenta la riga seguente per avere link in nero (per esempio per la stampa in bianco e nero)
    %colorlinks=false, linktocpage=false, pdfborder={0 0 0}, pdfstartpage=1, pdfstartview=FitV,
    breaklinks=true,
    pdfpagemode=UseNone,
    pageanchor=true,
    pdfpagemode=UseOutlines,
    plainpages=false,
    bookmarksnumbered,
    bookmarksopen=true,
    bookmarksopenlevel=1,
    hypertexnames=true,
    pdfhighlight=/O,
    %nesting=true,
    %frenchlinks,
    urlcolor=webbrown,
    linkcolor=RoyalBlue,
    citecolor=webgreen,
    %pagecolor=RoyalBlue,
    %urlcolor=Black, linkcolor=Black, citecolor=Black, %pagecolor=Black,
    pdftitle={\myTitle},
    pdfauthor={\textcopyright\ \myName, \myUni, \myFaculty},
    pdfsubject={},
    pdfkeywords={},
    pdfcreator={pdfLaTeX},
    pdfproducer={LaTeX}
}

%**************************************************************
% Impostazioni di itemize
%**************************************************************
%\renewcommand{\labelitemi}{$\ast$}

%\renewcommand{\labelitemi}{$\bullet$}
%\renewcommand{\labelitemii}{$\cdot$}
%\renewcommand{\labelitemiii}{$\diamond$}
%\renewcommand{\labelitemiv}{$\ast$}


%**************************************************************
% Impostazioni di listings
%**************************************************************
\lstset{
    language=[LaTeX]Tex,%C++,
    keywordstyle=\color{RoyalBlue}, %\bfseries,
    basicstyle=\small\ttfamily,
    %identifierstyle=\color{NavyBlue},
    commentstyle=\color{Green}\ttfamily,
    stringstyle=\rmfamily,
    numbers=none, %left,%
    numberstyle=\scriptsize, %\tiny
    stepnumber=5,
    numbersep=8pt,
    showstringspaces=false,
    breaklines=true,
    frameround=ftff,
    frame=single
} 


%**************************************************************
% Impostazioni di xcolor
%**************************************************************
\definecolor{webgreen}{rgb}{0,.5,0}
\definecolor{webbrown}{rgb}{.6,0,0}
\usepackage{chngcntr}
\counterwithout{footnote}{chapter}

%**************************************************************
% Altro
%**************************************************************

\newcommand{\omissis}{[\dots\negthinspace]} % produce [...]

% eccezioni all'algoritmo di sillabazione
\hyphenation
{
    ma-cro-istru-zio-ne
    gi-ral-din
}

\newcommand{\sectionname}{Section}
\addto\captions{\renewcommand{\figurename}{Figure}
                       \renewcommand{\tablename}{Table}}

\newcommand{\glsfirstoccur}{\ap{{[g]}}}

\newcommand{\intro}[1]{\emph{\textsf{#1}}}

%**************************************************************
% Environment per ``namespace description''
%**************************************************************

\newenvironment{namespacedesc}{
    \vspace{10pt}
    \par \noindent                              % start new paragraph
    \begin{description} 
}{
    \end{description}
    \medskip
}

\newcommand{\classdesc}[2]{\item[\textbf{#1:}] #2}
\renewcommand{\labelitemi}{$\bullet$}
\renewcommand{\labelitemii}{$\circ$}
\renewcommand{\labelitemiii}{-}
\renewcommand{\labelitemiv}{$\cdot$}
                     % file con le impostazioni personali
\raggedbottom
\begin{document}
%**************************************************************
% Materiale iniziale
%**************************************************************
    \frontmatter
    % !TEX encoding = UTF-8
% !TEX TS-program = pdflatex
% !TEX root = ../tesi.tex

%**************************************************************
% Frontespizio 
%**************************************************************
\begin{titlepage}

\begin{center}

\begin{LARGE}
\textbf{\myUni}\\
\end{LARGE}

\vspace{10pt}

\begin{Large}
\textsc{\myDepartment}\\
\end{Large}

\vspace{10pt}

\begin{large}
\textsc{\myFaculty}\\
\end{large}

\vspace{250pt}
%\begin{figure}[htbp]
%\begin{center}
%%\includegraphics[height=6cm]{logo}
%\end{center}
%\end{figure}
%\vspace{30pt}

\begin{LARGE}
\begin{center}
\textbf{\myTitle}\\
\end{center}
\end{LARGE}

\vspace{10pt} 

\begin{large}
\textsl{\myDegree}\\
\end{large}

\vspace{40pt}

\begin{large}
\begin{flushleft}
\textit{Relatore}\\ 
\vspace{5pt} 
\profTitle \myProf
\end{flushleft}

\vspace{0pt} 

\begin{flushright}
\textit{Laureando}\\ 
\vspace{5pt} 
\myName
\end{flushright}
\end{large}

\vspace{20pt}

\line(1, 0){338} \\
\begin{normalsize}
\textsc{Anno Accademico \myAA - Second session}
\end{normalsize}

\end{center}
    \end{titlepage}
    % !TEX encoding = UTF-8
% !TEX TS-program = pdflatex
% !TEX root = ../tesi.tex

%**************************************************************
% Colophon
%**************************************************************
\clearpage
\phantomsection
\thispagestyle{empty}

\hfill

\vfill

\noindent\myName: \textit{\myTitle,}
\myDegree,
\textcopyright\ \myTime.
    % !TEX encoding = UTF-8
% !TEX TS-program = pdflatex
% !TEX root = ../tesi.tex

%**************************************************************
% Dedica
%**************************************************************
\cleardoublepage
\phantomsection
\thispagestyle{empty}
\pdfbookmark{Dedica}{Dedica}

\vspace*{3cm}

\begin{center}
%    "Così come in algebra due affermazioni false ne danno una vera, così spero che il prodotto dei miei insuccessi si concluda con un successo." \medskip
%    --V. Van Gogh
\end{center}

\medskip

\begin{center}
Dedicated to my parents.
\end{center}

    % !TEX encoding = UTF-8
% !TEX TS-program = pdflatex
% !TEX root = ../tesi.tex

%**************************************************************
% Sommario
%**************************************************************
\cleardoublepage
\phantomsection
\pdfbookmark{Summary}{Summary}
\begingroup
\let\clearpage\relax
\let\cleardoublepage\relax
\let\cleardoublepage\relax

\chapter*{Summary}

%\vfill
%
%\selectlanguage{english}
%\pdfbookmark{Abstract}{Abstract}
%\chapter*{Abstract}
%
%\selectlanguage{italian}
This dissertation describes a deepening study about Visual Odometry problem tackled with transformer architectures.
The initial objectives were: create a synthetic dataset using BlenderProc2 framework, try different versions of transformer architectures which includes:
ResNet feature-extractor with encoder, ResNet feature-extractor with encoder-decoder, ResNet-feature extractor with encoder-decoder and pose Auto-encoder.


\endgroup			

\vfill


    % !TEX encoding = UTF-8
% !TEX TS-program = pdflatex
% !TEX root = ../tesi.tex

%**************************************************************
% Ringraziamenti
%**************************************************************
\cleardoublepage
\phantomsection
\pdfbookmark{Thanks}{Thanks}
\begin{flushright}{
	\slshape
	``Dio benedica quelle persone che quando incroci il loro sguardo per sbaglio, sorridono.''} \\
	\medskip
\end{flushright}


\bigskip

\begingroup
\let\clearpage\relax
\let\cleardoublepage\relax
\let\cleardoublepage\relax

\chapter*{Thanks}

\noindent \textit{Innanzitutto, vorrei esprimere la mia gratitudine al Prof. Sperduti, relatore della mia tesi, e Alessandro Proscia, il tutor aziendale, per l'aiuto e il sostegno fornitomi durante la stesura del lavoro.}\\

\noindent \textit{Desidero ringraziare con affetto i miei genitori per il sostegno, il grande aiuto e per essermi stati vicini in ogni momento durante gli anni di studio.}\\

\noindent \textit{Ho desiderio di ringraziare poi Veronica, Alberto, Marco, Lorenzo, Linpeng, Tommaso, Alessandro e Giulio per tutti i bellissimi anni passati insieme, le avventure vissute e di essersi sorbiti mille delle mie lamentele.}\\

\noindent \textit{Infine, vorrei esprimere la mia gratitudine alla famiglia Geminian e Bernardi per tutti gli aiuti ricevuti durante questi anni.}\\
\bigskip

\noindent\textit{\myLocation, \myTime}
\hfill \myName

\endgroup


    % !TEX encoding = UTF-8
% !TEX TS-program = pdflatex
% !TEX root = ../tesi.tex

%**************************************************************
% Indici
%**************************************************************
\cleardoublepage
\pdfbookmark{\contentsname}{tableofcontents}
\setcounter{tocdepth}{2}
\tableofcontents
%\markboth{\contentsname}{\contentsname} 
\clearpage

\begingroup 
    \let\clearpage\relax
    \let\cleardoublepage\relax
    \let\cleardoublepage\relax
    %*******************************************************
    % Elenco delle figure
    %*******************************************************    
    \phantomsection
    \pdfbookmark{\listfigurename}{lof}
    \listoffigures

    \vspace*{8ex}

    %*******************************************************
    % Elenco delle tabelle
    %*******************************************************
    \phantomsection
    \pdfbookmark{\listtablename}{lot}
    \listoftables
    \vspace*{8ex}
\endgroup

\cleardoublepage

    \cleardoublepage

%**************************************************************
% Materiale principale
%**************************************************************
    \mainmatter
    % !TEX encoding = UTF-8
% !TEX TS-program = pdflatex
% !TEX root = ../tesi.tex

%**************************************************************
\chapter{Introduzione}\label{ch:introduzione}
%**************************************************************

\intro{In questa sezione viene svolta una breve introduzione alle idee dello stage e una breve presentazione dell'azienda Imola Informatica S.p.A.}
%\noindent Esempio di utilizzo di un termine nel glossario \\
%\gls{api}. \\
%
%\noindent Esempio di citazione in linea \\
%\cite{site:agile-manifesto}. \\
%
%\noindent Esempio di citazione nel pie' di pagina \\
%citazione\footcite{womak:lean-thinking} \\

%\input{capitoli/1_1_idea}
%\input{capitoli/1_2_introduzione_al_progetto}
%\input{capitoli/1_3_azienda}
%\input{capitoli/1_4_risultati.tex}
%\input{capitoli/1_5_organizzazione_del_testo.tex}
             % Introduzione
    % !TEX encoding = UTF-8
% !TEX TS-program = pdflatex
% !TEX root = ../tesi.tex

%**************************************************************
\chapter{Internship Description}
\label{ch:internship-description}
%**************************************************************

\intro{In this chapter will be described the internship, the partition of the tasks during the whole duration of the internship.}\\

%\input{./chapters/2_1_modalita_svolgimento.tex}
%
%\input{./chapters/2_2_analisi_dei_rischi.tex}
%
%\input{./chapters/2_3_requisiti_obiettivi.tex}
%
%\input{./chapters/2_4_pianificazione.tex}             % Kick-Off
    % !TEX encoding = UTF-8
% !TEX TS-program = pdflatex
% !TEX root = ../tesi.tex

%**************************************************************
\chapter{Datasets}
\label{ch:datasets}
%**************************************************************

\intro{
In this chapter will be presented dataset and synthetic dataset created.
}\\

%\input{./capitoli/3_1_casi_uso.tex}
%\input{./capitoli/3_2_requisiti.tex}
%\input{./capitoli/3_3_tracciamento_requisiti.tex}



             % Concept Preview
    % !TEX encoding = UTF-8
% !TEX TS-program = pdflatex
% !TEX root = ../tesi.tex

%**************************************************************

\chapter{Theoretical foundations}
\label{ch:theoretical-foundations}
%**************************************************************

\intro{In this chapter will be presented the main theoretical knowledge useful to understand the content from successive chapters.}\\

\input{./chapters/3_1_deep_learning.tex}
\section{Visual Odometry}\label{sec:visual-odometry}

Visual Odometry is an important task in robotics' computer vision field, because it allows the robot to understand where it is and how it is oriented.
%\input{./chapters/4_3_progettazione.tex}
%\input{./chapters/4_4_design_pattern_utilizzati.tex}
%\input{./chapters/4_5_codifica.tex}             % Product Prototype
    % !TEX encoding = UTF-8
% !TEX TS-program = pdflatex
% !TEX root = ../tesi.tex

%**************************************************************
\chapter{Implementations}
\label{ch:implementations}
\intro{In this chapter will be presented the different implementations of the transformer model.}

%%**************************************************************
%
%\input{./capitoli/5_1_analisi_statica.tex}
%
%%**************************************************************
%
%\input{./capitoli/5_2_test_unitari.tex}
%
%%**************************************************************
%
%\input{./capitoli/5_3_test_integrazione}
%
%%**************************************************************
%
%\input{./capitoli/5_4_test_sistema}
%
%%**************************************************************
             % Product Design Freeze e SOP
    % !TEX encoding = UTF-8
% !TEX TS-program = pdflatex
% !TEX root = ../tesi.tex

%**************************************************************


\chapter{Final discussions}
\label{ch:final-discussions}
%**************************************************************
\intro{In this chapter will be discussed the results achieved.}

%\input{./chapters/6_1_consuntivo_finale.tex}
%
%\input{./chapters/6_2_raggiungimento_obiettivi.tex}
%
%\input{./chapters/6_3_prodotti_ottenuti.tex}
%
%\input{./chapters/6_4_conoscenze_acquisite.tex}
%
%\input{./chapters/6_5_valutazione_personale.tex}             % Conclusioni
%**************************************************************
% Materiale finale
%**************************************************************
    \backmatter
    \printglossaries
    % !TEX encoding = UTF-8
% !TEX TS-program = pdflatex
% !TEX root = ../tesi.tex

%**************************************************************
% Bibliografia
%**************************************************************

\cleardoublepage
\chapter{Bibliopraphy}\label{ch:bibliografia}

\nocite{*}
% Stampa i riferimenti bibliografici
\printbibliography[heading=subbibliography,title={Riferimenti bibliografici},type=book]

% Stampa i siti web consultati
\printbibliography[heading=subbibliography,title={Siti web consultati},type=online]
\printbibliography[heading=subbibliography,title={Articoli consultati},type=article]


\end{document}
