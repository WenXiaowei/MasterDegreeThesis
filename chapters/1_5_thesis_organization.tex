\section{Thesis Organization}\label{sec:thesis-organization}
\begin{description}
    \item[{\hyperref[ch:introduction]{First chapter}}] introduces the general content about thesis and gives a short presentation of the project and the results;

    \item[{\hyperref[ch:theoretical-foundations]{Second chapter}}] a deepening about the theoretical foundations used during the stage and the project;

    \item[{\hyperref[ch:datasets]{Third chapter}}] presents the datasets used during for the training and the testing of the model;

    \item[{\hyperref[ch:state-of-the-art]{Fourth chapter}}] presents the state-of-the-art of Visual Odometry;

    \item[{\hyperref[ch:experiments]{Fifth chapter}}] presents the experiments did during to develop the system;

    \item[{\hyperref[ch:implementations]{Sixth chapter}}] presents the different implementations of the system;

    \item[{\hyperref[ch:final-discussions]{Seventh chapter}}] discusses about the results and possible future developments.
\end{description}
During the drafting of the essay, following typography conventions are considered:
\begin{itemize}
    \item the acronyms, abbreviations, ambiguous terms or terms not in common use are defined in the glossary, in the end of the present document;
    \item the first occurrences of the terms in the glossary are highlighted like this: \gls{word};
    \item the terms from the foreign language or jargon are highlighted like this: \emph{italics}.
\end{itemize}