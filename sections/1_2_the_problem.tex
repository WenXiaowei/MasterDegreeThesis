\section{Problem}\label{sec:problem}
The term "odometry" originated from two Greek works \emph{hodos} (meaning "journery" or "travel") and \emph{metron} (meaning "measure").
This derivation is related to the estimation of the change in a robot's pose (translation and rotation) over time.
Mobile robot use data from motion sensor to estimate their position relative to their initial location, this is called odometry.
VO is a technique used to localize a robot by using only a stream of images acquired from a single or multiple camera.
There are different ways to classify the typology of Visual Odometry:
\begin{itemize}
    \item based on the camera setup:
        \begin{itemize}
            \item Monocular VO: using only one camera;
            \item Stereo VO: using two cameras;
        \end{itemize}
    \item based on the information:
        \begin{itemize}
            \item Feature based method: which extracts the image feature points and tracks them in the image sequence;
            \item Direct method: a novel method which uses the pixel intensity in the image sequence directly as visual input.
            \item Hybrid method: which combines the two methods.
        \end{itemize}
    \item Visual inertial odometry: if a \gls{imu} is used within the VO system, it is commonly referred to as Visual inertial odometry.
\end{itemize}
We can represent the pose in different ways, for example: \textbf{euler angles}, \textbf{quaternions}, \textbf{SO(3)}, \textbf{rotation matrices} combined with \textbf{translation vectors}.

The goal is to create a \gls{nn}, using a \textbf{ResNet} to extract features from images and the \textbf{transformer} presented by ~\cite{transformer_paper}, which is able to estimate a sequence of camera's pose given a sequence of images.