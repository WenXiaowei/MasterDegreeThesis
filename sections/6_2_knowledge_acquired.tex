\section{Knowledge acquired}\label{sec:knowledge-acquired}
By developing this project, I amplified my knowledge with various topics, such as:
\begin{itemize}
    \item Deep Learning,
    \item Computer Vision.
    \item Anaconda.
    \item PyTorch.
    \item Transformers.
    \item Auto-regressive models.
\end{itemize}
Especially, how the transformer works, and how to use it for computer vision tasks.
At the beginning of the project, the word ``transformer'' has had other meanings, but now, I know that it is a neural network architecture that is able to learn the sequence of images and the sequence of poses in a self-supervised way.
It's a general purpose architecture for domain adaptation from one sequence to another.
Another important notion that I learned is the concept of auto-regressive models.

To develop this approach, I had to deepen my knowledge about Visual Odometry, especially, a solid understanding of the task, basis knowledge such as the pose representations and the conversion between them.
I also had to amply my knowledge about the deep-learning framework, PyTorch, specifically, how to build a model, which are the components that are provided by the library, and how to use them.
For example, I had to implement the Transformer, initially I was implementing it from scratch, but then I discovered that there was some classes already implemented in the library, so I used latter ones, because they are more efficient and they are already tested.
I had to use Anaconda to manage the Python environment, because it is a very useful tool, and it is very easy to use.
